\documentclass{article}
\usepackage[utf8]{inputenc}

\title{main.tex}
\author{t190063 }
\date{April 2020}

\begin{document}
\maketitle
\section{Introduction}
まず、底面は円なので、底面積はそのまま、
πr^2で求めることが出来ますね。
円錐の側面を展開すると、扇形になります。ここで、図に赤線で示した「扇形の弧」と「底面の円周」は、もともと接していたため、長さが等しいことに注目します。つまり、底面の円周は 
2πr2πrなので、扇形の弧の長さも 2r2πrになります。
最後に、扇形の面積は弧の長さに比例することを用います。半径 R の円周は 2πR2πR面積は πR2です。この円のうち、弧の長さ 
2πr2πrとなる扇形の面積を求めればよいことが分かります。つまり、半径 R の円の面積に、「円周に対する弧の長さの割合」を掛ければよいのです。計算すると、扇形の面積=πR2×2πr2πR=πRr扇形の面積=πR2×2πr2πR=πRrとなり、側面積が求まりました。よって円錐の側面積 Sは、底面積と側面積を足し合わせて、S=πr2+πRr=πr(r+R)S=πr2+πRr=πr(r+R)となります。これが円錐の表面積を求める公式です。
\end{document} 
